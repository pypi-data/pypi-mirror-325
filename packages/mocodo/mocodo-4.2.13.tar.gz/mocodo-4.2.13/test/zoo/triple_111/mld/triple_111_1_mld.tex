\documentclass[a4paper]{article}
\usepackage[normalem]{ulem}
\usepackage[T1]{fontenc}
\usepackage[french]{babel}
\frenchsetup{StandardLayout=true}

\newcommand{\relat}[1]{\textsc{#1}}
\newcommand{\attr}[1]{#1}
\newcommand{\prim}[1]{\uline{#1}}
\newcommand{\foreign}[1]{\#\textsl{#1}}

\title{Conversion en relationnel\\du MCD \emph{triple\_111}}
\author{\emph{Généré par Mocodo}}

\begin{document}
\maketitle

\begin{itemize}
  \item \relat{Projet} (\prim{projet}, \attr{libellé})
  \begin{itemize}
    \item Le champ \emph{projet} constitue la clé primaire de la table. C'était déjà un identifiant de l'entité \emph{Projet}.
    \item Le champ \emph{libellé} était déjà un simple attribut de l'entité \emph{Projet}.
  \end{itemize}

  \item \relat{Technicien} (\prim{technicien}, \attr{nom technicien})
  \begin{itemize}
    \item Le champ \emph{technicien} constitue la clé primaire de la table. C'était déjà un identifiant de l'entité \emph{Technicien}.
    \item Le champ \emph{nom technicien} était déjà un simple attribut de l'entité \emph{Technicien}.
  \end{itemize}

  \item \relat{Utiliser} (\prim{carnet}$^{u\_1}$, \foreign{\prim{projet}}$^{u\_2}$, \foreign{technicien}$^{u\_1 u\_2}$)
  \begin{itemize}
    \item Le champ \emph{carnet} fait partie de la clé primaire de la table. Sa table d'origine (\emph{Carnet}) ayant été supprimée, il n'est pas considéré comme clé étrangère. Il obéit par contre à la contrainte d'unicité 1.
    \item Le champ \emph{projet} fait partie de la clé primaire de la table. C'est une clé étrangère qui a migré directement à partir de l'entité \emph{Projet}. Il obéit en outre à la contrainte d'unicité 2.
    \item Le champ \emph{technicien} est une clé étrangère. Il a migré directement à partir de l'entité \emph{Technicien} en perdant son caractère identifiant. Il obéit en outre aux contraintes d'unicité 1 et 2.
  \end{itemize}

\end{itemize}

\end{document}
