\documentclass[a4paper]{article}
\usepackage[normalem]{ulem}
\usepackage[T1]{fontenc}
\usepackage[french]{babel}
\frenchsetup{StandardLayout=true}

\newcommand{\relat}[1]{\textsc{#1}}
\newcommand{\attr}[1]{#1}
\newcommand{\prim}[1]{\uline{#1}}
\newcommand{\foreign}[1]{\#\textsl{#1}}

\title{Conversion en relationnel\\du MCD \emph{inheritance\_weak}}
\author{\emph{Généré par Mocodo}}

\begin{document}
\maketitle

\begin{itemize}
  \item \relat{CONTRAT} (\foreign{\prim{num prof}}, \prim{date contrat}, \attr{salaire horaire contrat})
  \begin{itemize}
    \item Le champ \emph{num prof} fait partie de la clé primaire de la table. C'est une clé étrangère qui a migré à partir de l'entité \emph{VACATAIRE} pour renforcer l'identifiant.
    \item Le champ \emph{date contrat} fait partie de la clé primaire de la table. C'était déjà un identifiant de l'entité \emph{CONTRAT}.
    \item Le champ \emph{salaire horaire contrat} était déjà un simple attribut de l'entité \emph{CONTRAT}.
  \end{itemize}

  \item \relat{PROFESSEUR} (\prim{num prof}, \attr{nom prof}, \attr{prénom prof}, \attr{téléphone prof})
  \begin{itemize}
    \item Le champ \emph{num prof} constitue la clé primaire de la table. C'était déjà un identifiant de l'entité \emph{PROFESSEUR}.
    \item Les champs \emph{nom prof}, \emph{prénom prof} et \emph{téléphone prof} étaient déjà de simples attributs de l'entité \emph{PROFESSEUR}.
  \end{itemize}

  \item \relat{SALARIÉ} (\foreign{\prim{num prof}}, \attr{date embauche salarié}, \attr{échelon salarié}, \attr{salaire salarié})
  \begin{itemize}
    \item Le champ \emph{num prof} constitue la clé primaire de la table. C'est une clé étrangère qui a migré à partir de l'entité-mère \emph{PROFESSEUR}.
    \item Les champs \emph{date embauche salarié}, \emph{échelon salarié} et \emph{salaire salarié} étaient déjà de simples attributs de l'entité \emph{SALARIÉ}.
  \end{itemize}

  \item \relat{VACATAIRE} (\foreign{\prim{num prof}}, \attr{statut vacataire})
  \begin{itemize}
    \item Le champ \emph{num prof} constitue la clé primaire de la table. C'est une clé étrangère qui a migré à partir de l'entité-mère \emph{PROFESSEUR}.
    \item Le champ \emph{statut vacataire} était déjà un simple attribut de l'entité \emph{VACATAIRE}.
  \end{itemize}

\end{itemize}

\end{document}
