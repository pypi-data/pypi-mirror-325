\documentclass[a4paper]{article}
\usepackage[normalem]{ulem}
\usepackage[T1]{fontenc}
\usepackage[french]{babel}
\frenchsetup{StandardLayout=true}

\newcommand{\relat}[1]{\textsc{#1}}
\newcommand{\attr}[1]{#1}
\newcommand{\prim}[1]{\uline{#1}}
\newcommand{\foreign}[1]{\#\textsl{#1}}

\title{Conversion en relationnel\\du MCD \emph{alt}}
\author{\emph{Généré par Mocodo}}

\begin{document}
\maketitle

\begin{itemize}
  \item \relat{CLIENT} (\prim{Réf. client}, \attr{Nom}$^{u\_1}$, \attr{Prénom}$^{u\_1}$, \attr{Adresse}, \attr{Mail}$^{u\_2}$)
  \begin{itemize}
    \item Le champ \emph{Réf. client} constitue la clé primaire de la table. C'était déjà un identifiant de l'entité \emph{CLIENT}.
    \item Les champs \emph{Nom} et \emph{Prénom} étaient déjà de simples attributs de l'entité \emph{CLIENT}. Il obéit à la contrainte d'unicité 1.
    \item Le champ \emph{Adresse} était déjà un simple attribut de l'entité \emph{CLIENT}.
    \item Le champ \emph{Mail} était déjà un simple attribut de l'entité \emph{CLIENT}. Il obéit à la contrainte d'unicité 2.
  \end{itemize}

  \item \relat{FOO} (\prim{foo}, \attr{bar}$^{u\_1}$, \attr{biz}$^{u\_1 u\_2}$, \attr{buz}$^{u\_2}$, \attr{qux}$^{u\_3}$, \attr{quux}$^{u\_1 u\_2 u\_3}$)
  \begin{itemize}
    \item Le champ \emph{foo} constitue la clé primaire de la table. C'était déjà un identifiant de l'entité \emph{FOO}.
    \item Le champ \emph{bar} était déjà un simple attribut de l'entité \emph{FOO}. Il obéit à la contrainte d'unicité 1.
    \item Le champ \emph{biz} était déjà un simple attribut de l'entité \emph{FOO}. Il obéit aux contraintes d'unicité 1 et 2.
    \item Le champ \emph{buz} était déjà un simple attribut de l'entité \emph{FOO}. Il obéit à la contrainte d'unicité 2.
    \item Le champ \emph{qux} était déjà un simple attribut de l'entité \emph{FOO}. Il obéit à la contrainte d'unicité 3.
    \item Le champ \emph{quux} était déjà un simple attribut de l'entité \emph{FOO}. Il obéit aux contraintes d'unicité 1, 2 et 3.
  \end{itemize}

  \item \relat{UTILISER} (\prim{carnet}$^{u\_1}$, \prim{projet}$^{u\_2}$, \attr{technicien}$^{u\_1 u\_2}$)
  \begin{itemize}
    \item Le champ \emph{carnet} fait partie de la clé primaire de la table. C'était déjà un identifiant de l'entité \emph{UTILISER}. Il obéit en outre à la contrainte d'unicité 1.
    \item Le champ \emph{projet} fait partie de la clé primaire de la table. C'était déjà un identifiant de l'entité \emph{UTILISER}. Il obéit en outre à la contrainte d'unicité 2.
    \item Le champ \emph{technicien} était déjà un simple attribut de l'entité \emph{UTILISER}. Il obéit aux contraintes d'unicité 1 et 2.
  \end{itemize}

\end{itemize}

\end{document}
