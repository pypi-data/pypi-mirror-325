\documentclass{article}
% common styling and macros shared by all proof files

\usepackage[top=1in, right=1in, left=1in, bottom=1.5in]{geometry}

\usepackage{amsmath,amsthm,amsfonts,amssymb,amscd}
\usepackage{listings}
\usepackage{hyperref}
\usepackage{xcolor}
\usepackage{xr}

\usepackage{enumerate} 
\usepackage{physics}
\usepackage{fancyhdr}
\usepackage{hyperref}
\usepackage{graphicx}
\usepackage{tcolorbox}
\usepackage{catchfile}
\usepackage{pdftexcmds}
\usepackage[T1]{fontenc}

% hyperref
\hypersetup{
  colorlinks=true,
  linkcolor=blue,
  linkbordercolor={0 0 1}
}

% \contrib macro to indicate inclusion in "contrib".
\usepackage{tcolorbox}
\newtcolorbox{warn_box}{colback=red!5!white,colframe=red!75!black}
\newcommand{\contrib}{{\begin{warn_box}This proof resides in \textbf{``contrib''} because it has not completed the vetting process.\end{warn_box}}} 
\newcommand{\floatingPoint}{{\begin{warn_box}This implementation is susceptible to floating-point vulnerabilities.\end{warn_box}}} 

% asOfCommit macro to version a code dependency. Arguments:
%    #1: relative path to file you are dependent on
%    #2: commit hash it was last edited. If outdated, this should be the second hash in the footnoote. Otherwise,
%            git log -n 1 --pretty=format:%h -- path/to/file.rs
\makeatletter
\ifnum\pdf@shellescape=1
   % "private" command that builds a link to a blob
  \newcommand{\linkOpendpBlob}[3]{%
    \href{https://github.com/opendp/opendp/blob/#1/#2#3}{\path{#3} at commit #1}}

  % latex macro expansion has a separate phase for \input evaluation
  %     immediately evaluate a command to write a temp file to ./out containing the current directory
  \immediate\write18{[ ! -f out/cwd.txt ] && (mkdir -p out && git rev-parse --show-prefix | sed "s|_|\@backslashchar\@backslashchar\@backslashchar_|g" > out/cwd.txt)}
  %     ...and then retrieve the current working directory by loading the temp file
  \CatchFileDef\GitWorkingDir{out/cwd.txt}{\endlinechar=-1}

  % command for building the (up to date) or (outdated) status
  \newcommand{\fileStatus}[2]{%
  \setbox0=\hbox{\input|"git --no-pager log -n1 --pretty='\@percentchar H' #1 | grep -E '^#2.*'"\unskip}\ifdim\wd0=0pt
        (outdated\footnote{See new changes with \texttt{git diff #2..\input|"git --no-pager log -n1 --pretty='\@percentchar h' #1" \GitWorkingDir\path{#1}}})\else
        (up to date)\fi
  }

  \newcommand{\asOfCommit}[2]{%
      % permalink the target
      \linkOpendpBlob{#2}{\GitWorkingDir}{#1}
      % conditionally add (outdated) or (up to date) depending on matching commit hash
      \fileStatus{#1}{#2}%
  }
\else
  % simplified command if shell-escape not enabled
  \newcommand{\asOfCommit}[2]{#1 at commit #2 (unknown status\footnote{Shell-escape is not enabled. Enable \texttt{--shell-escape} to check if this proof is up-to-date with the code.})}
\fi
\makeatother

% \vettingPR macro to link a PR. Arguments:
%    #1: PR number
\newcommand{\vettingPR}[1]{\href{https://github.com/opendp/opendp/pull/#1}{Pull Request \##1}}

% for links to rustdoc items in OpenDP. Arguments:
%    #1: path to item, and designation as trait, struct, fn, etc.
%    #2: item name
\makeatletter
\ifnum\pdf@shellescape=1
  % latex macro expansion has a separate phase for \input evaluation
  %     immediately evaluate a command to write a temp file to ./out containing the base path
  \immediate\write18{[ ! -f out/rustdoc.txt ] && mkdir -p out && ([ -z `kpsewhich --var-value OPENDP_RUSTDOC_PORT` ] && echo "https://docs.rs/opendp/`head -n 1 \@backslashchar`git rev-parse --show-toplevel\@backslashchar`/VERSION | sed 's|.*-dev.*|latest|g'`" || echo "http://localhost:`kpsewhich --var-value OPENDP_RUSTDOC_PORT`") > out/rustdoc.txt}
  %     ...and then retrieve the base path by loading the temp file
  \CatchFileDef\OpenDPRustdocBase{out/rustdoc.txt}{\endlinechar=-1}
\else
  % if shell commands are not enabled, just claim latest
  \newcommand{\OpenDPRustdocBase}{https://docs.rs/opendp/latest}
\fi
\makeatother
\newcommand{\rustdoc}[2]{\href{\OpenDPRustdocBase/opendp/#1.#2.html}{\texttt{#2}}}

% for links to external dependencies. Arguments:
%    #1: crate name
%    #2: path to item, and designation as trait, struct, fn, etc.
%    #3: item name
\newcommand{\docsrs}[3]{\href{https://docs.rs/#1/latest/#1/#2.#3.html}{\texttt{#3}}}

% minted (pseudocode)
\definecolor{codegreen}{rgb}{0,0.6,0}
\definecolor{codegray}{rgb}{0.5,0.5,0.5}
\definecolor{codepurple}{rgb}{0.58,0,0.82}
\definecolor{backcolour}{rgb}{0.95,0.95,0.92}

\lstdefinestyle{mystyle}{
    backgroundcolor=\color{backcolour},   
    commentstyle=\color{codegreen},
    keywordstyle=\color{magenta},
    numberstyle=\tiny\color{codegray},
    stringstyle=\color{codepurple},
    basicstyle=\ttfamily\footnotesize,
    breakatwhitespace=false,         
    breaklines=true,                 
    captionpos=b,                    
    keepspaces=true,                 
    numbers=left,                    
    numbersep=5pt,                  
    showspaces=false,                
    showstringspaces=false,
    showtabs=false,                  
    tabsize=2
}

\lstset{style=mystyle}

% common commands
\theoremstyle{definition}
\newtheorem{theorem}{Theorem}[section]
\newtheorem{lemma}[theorem]{Lemma}
\newtheorem{definition}[theorem]{Definition}
\newtheorem{warning}{Warning}
\newtheorem{corollary}{Corollary}
\newtheorem{proposition}{Proposition}
\newtheorem{remark}{Remark}
\newtheorem{observation}{Observation}
\newtheorem{note}{Note}

\newcommand{\vicki}[1]{{ {\color{olive}{(vicki)~#1}}}}
\newcommand{\hanwen}[1]{{ {\color{purple}{(hanwen)~#1}}}}
\newcommand{\zach}[1]{{ {\color{red}{(zach)~#1}}}}

\newcommand{\MultiSet}{\mathrm{MultiSet}}
\newcommand{\len}{\mathrm{len}}
\newcommand{\din}{\texttt{d\_in}}
\newcommand{\dout}{\texttt{d\_out}}
\newcommand{\T}{\texttt{T} }
\newcommand{\F}{\texttt{F} }
\newcommand{\Map}{\texttt{Map}}
\newcommand{\X}{\mathcal{X}}
\newcommand{\Y}{\mathcal{Y}}
\newcommand{\True}{\texttt{True}}
\newcommand{\False}{\texttt{False}}
\newcommand{\clamp}{\texttt{clamp}}
\newcommand{\function}{\texttt{function}}
\newcommand{\float}{\texttt{float }}
\newcommand{\questionc}[1]{\textcolor{red}{\textbf{Question:} #1}}


\newcommand{\validTransformation}[2]{%
  \begin{theorem}
  For every setting of the input parameters #1 to #2 such that the given preconditions
  hold, #2 raises an exception (at compile time or run time) or returns a valid transformation. A valid transformation has the following properties:
  \begin{enumerate}
      \item \textup{(Appropriate output domain).} 
      For every element $x$ in \texttt{input\_domain}, $\function(x)$ is in \texttt{output\_domain} or raises a data-independent runtime exception.
      
      \item \textup{(Stability guarantee).} 
      For every pair of elements $x, x'$ in \texttt{input\_domain} and for every pair $(\din, \dout)$, 
      where \din\ has the associated type for \texttt{input\_metric} and \dout\ has the associated type for \\ \texttt{output\_metric}, 
      if $x, x'$ are \din-close under \texttt{input\_metric}, $\texttt{stability\_map}(\din)$ does not raise an exception,
      and $\texttt{stability\_map}(\din) \leq \dout$, 
      then $\function(x), \function(x')$ are $\dout$-close under \texttt{output\_metric}.
  \end{enumerate}
  \end{theorem}
}


\newcommand{\validMeasurement}[2]{%
  \begin{theorem}
  For every setting of the input parameters #1 to #2 such that the given preconditions
  hold, #2 raises an exception (at compile time or run time) or returns a valid measurement. A valid measurement has the following property:
  \begin{enumerate}
      \item \textup{(Privacy guarantee).}
      For every pair of elements $x, x'$ in \texttt{input\_domain} and for every pair $(\din, \dout)$,
      where \din\ has the associated type for \texttt{input\_metric} and \dout\ has the associated type for \\ \texttt{output\_measure},
      if $x, x'$ are \din-close under \texttt{input\_metric}, $\texttt{privacy\_map}(\din)$ does not raise an exception,
      and $\texttt{privacy\_map}(\din) \leq \dout$,
      then $\function(x), \function(x')$ are $\dout$-close under \texttt{output\_measure}.
  \end{enumerate}
  \end{theorem}
}




\title{\texttt{fn make\_randomized\_response}}
\author{Michael Shoemate}
\begin{document}

\maketitle

\contrib

Proves soundness of \rustdoc{measurements/fn}{make\_randomized\_response} in \asOfCommit{mod.rs}{f5bb719},
a constructor taking a category set \texttt{categories} and probability \texttt{prob}.
The mechanism returned by \texttt{make\_randomized\_response} takes in a data set \texttt{arg} (a single category), and...

\begin{itemize}
    \item ...if \texttt{arg} is in \texttt{categories},
    the mechanism truthfully returns the same value \texttt{arg} with probability \texttt{prob},
    otherwise it lies by selecting one of the other categories uniformly at random.
    \item ...if \texttt{arg} is not in \texttt{categories}, 
    it returns a category chosen uniformly at random.
\end{itemize}

\subsection*{PR History}
\begin{itemize}
    \item \vettingPR{490}
\end{itemize}

\section{Hoare Triple}

\subsection*{Preconditions}
\begin{itemize}
    \item Variable \texttt{categories} must be a set with members of type \texttt{T}
    \item Variable \texttt{prob} must be of type \texttt{QO}
    \item Type \texttt{QO} must have trait \rustdoc{traits/trait}{Float}
    \item The bit representation of type \texttt{QO} must support \texttt{ExactIntCast} to and from \texttt{usize}
\end{itemize}

\subsection*{Pseudocode}
\lstinputlisting[language=Python,firstline=2,escapechar=|]{./pseudocode/make_randomized_response.py}

\subsection*{Postcondition}

\validMeasurement{\texttt{(categories, prob, T, QO)}}{\\ \texttt{make\_randomized\_response}}

\section{Proof}

\begin{proof} 
\textbf{(Privacy guarantee.)} 
    
\begin{tcolorbox}
    The proof assumes the following lemma.
    \begin{lemma}
        \texttt{sample\_uniform\_int\_below} and \texttt{sample\_bernoulli\_float} satisfy their postconditions.
    \end{lemma}
\end{tcolorbox}

\texttt{sample\_uniform\_int\_below} and \texttt{sample\_bernoulli\_float} can only fail due to lack of system entropy. 
This is usually related to the computer's physical environment and not the dataset. 
The rest of this proof is conditioned on the assumption that these functions do not raise an exception. 

Let $x$ and $x'$ be datasets that are \texttt{d\_in}-close with respect to \texttt{input\_metric}.
Here, the metric is \texttt{DiscreteMetric} which enforces that $\din \geq 1$ if $x \ne x'$ and $\din = 0$ if $x = x'$. 
If $x = x'$, then the output distributions on $x$ and $x'$ are identical, and therefore the max-divergence is 0.

Now consider the case where $x \ne x'$. 
For shorthand, we let $p$ represent \texttt{prob}, the probability of returning the input,
and $t$ denote the number of categories.
Note that all categories must be unique as the input data type is a set.
This means duplicate categories cannot influence the output distribution.

$t$ must be at least two, by pseudocode line \ref{line:num_cats}, as any fewer would not be useful.
$p$ is restricted to $[1/t, 1.0)$ by pseudocode line \ref{line:range}, as any less would not be useful.

We'll first consider all possible output probabilities, 
and then use this to upper bound the ratio of any two probabilities.
For any outcome $y \in \texttt{candidates}$, 
the probability of observing $y$ is one of three values:

\begin{enumerate}
    \item When the mechanism is honest: 
    \[
        \Pr[M(x) = y | y = x] = p
    \]

    \item When the mechanism lies: 
    \[
        \Pr[M(x) = y | y \ne x \wedge x \in \texttt{candidates}] = \frac{1 - p}{t - 1}
    \]

    \item When the input is not in the category set, the output is uniformly sampled from the candidates: 
    \[
        \Pr[M(x) = y | y \ne x \wedge x \not\in \texttt{candidates}] = \frac{1}{t}
    \]
\end{enumerate}

\begin{tcolorbox}
\begin{lemma}
    \label{bounded-case-3}
    The probability of case 3 is bounded by cases one and two:
     \begin{equation}
        \frac{1 - p}{t - 1} \leq \frac{1}{t} \leq p
     \end{equation}
\end{lemma}

\begin{proof}
$1 / t$ is bounded above by case one ($p$) due to pseudocode line \ref{line:range}. 
Reusing \ref{line:range}, $\frac{1 - p}{t - 1} \leq \frac{1 - 1/t}{t - 1} = \frac{1}{t}$.
Therefore $1 / t$ is also bounded below by case two ($\frac{1 - p}{t - 1}$).
\end{proof}
\end{tcolorbox}

By \ref{bounded-case-3}, the divergence is never maximized when the input is not in the category set,
which simplifies the following analysis.

We now consider the max-divergence of the mechanism over all choices of neighboring datasets.
    
\begin{align}
    &\max_{x \sim x'} D_{\infty}(M(x), M(x')) \\
    =& \max_{x \sim x'} \max_{S \subseteq Supp(M(x))}\ln (\frac{\Pr[M(x) \in S]}{\Pr[M(x') \in S]}) \\
    \le& \max_{x \sim x'} \max_{y \in Supp(M(x))}\ln (\frac{\Pr[M(x) = y]}{\Pr[M(x') = y]}) &&\text{Lemma 3.3 } \cite{Kasiviswanathan_2014} \\
    =& \ln \left(\max\left(\frac{p \cdot (t - 1)}{1 - p}, \frac{(1 - p) \cdot (t - 1)}{p}, \frac{(1 - p) \cdot (t - 1)}{(1 - p) \cdot (t - 1)}\right)\right) \label{max-terms} \\
    =& \ln (\frac{p \cdot (t - 1)}{1 - p})
\end{align}

The terms in the maximum on line \ref{max-terms} cover all combinations of $x$, $x'$ and $y$. Respectively:
\begin{enumerate}
    \item When $y = x$.
    \item When $y \ne x$ and $y = x'$.
    \item When $y \ne x$ and $y \ne x'$.
\end{enumerate}

Pseudocode line \ref{line:map} implements this bound with conservative rounding towards positive infinity. 
When $\din > 0$ and no exception is raised in computing $\texttt{c} = \texttt{privacy\_map}(\din)$, then $\ln\left(\frac{p \cdot (t - 1)}{1 - p}\right) \leq \texttt{c}$. 

Therefore it has been shown that for every pair of elements $x, x' \in \texttt{input\_domain}$ and every $d_{DM}(x, x') \le \din$ with $\din \ge 0$, 
if $x, x'$ are $\din$-close then $\function(x),\function(x')$ are $\texttt{privacy\_map}(\din)$-close under $\texttt{output\_measure}$ (the Max-Divergence).
\end{proof}

\bibliographystyle{plain}
\bibliography{randomized_response.bib}

\end{document}
